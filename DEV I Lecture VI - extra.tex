\documentclass{beamer}
\usetheme[hideothersubsections]{HRTheme}
\usepackage{beamerthemeHRTheme}
\usepackage[utf8]{inputenc}
\usepackage{graphicx}
\usepackage[space]{grffile}
\usepackage{listings}
\lstset{language=Python,
basicstyle=\ttfamily\footnotesize,
mathescape=true,
frame=single,
breaklines=true}
\lstset{
  literate={ï}{{\"i}}1
           {ì}{{\`i}}1
}
\usepackage{color}
\newcommand{\red}[1]{
\textcolor{red}{#1}
}
\newcommand{\ts}{\textbackslash}

\title{Looping with \texttt{for}}

\author{The INFDEV Team @ HR}

\institute{Hogeschool Rotterdam \\ 
Rotterdam, Netherlands}

\date{}

\begin{document}
\maketitle

\SlideSection{Introduction}
\SlideSubSection{Lecture topics}
\begin{slide}{
\item the (lack of) limitations of \texttt{while} loops
\item \texttt{for} statements and their semantics
\item \texttt{for} as a \textit{limited} form of \texttt{while}
}\end{slide}

\SlideSection{\texttt{while} loops}
\SlideSubSection{Potential issues}
\begin{slide}{
\item While loops specify unbounded iteration
\item This means that the number of iterations is not necessarily easy to specify
\item For example
\begin{itemize}
\item Virtual machines
\item User-driven loops
\item Servers
\item Operating systems
\item ...
\end{itemize}
}\end{slide}

\begin{frame}[fragile]{Unbounded loop example}
\begin{lstlisting}
n,m = input("Let's have two numbers")
cnt = 1
while n > m:
  cnt = cnt + 1
  n = n / m
print("Result is %d" % cnt)
\end{lstlisting}

\textbf{What does this code do?}

\textbf{How many steps does it take?}
\end{frame}

\begin{frame}[fragile]{Unbounded loop example}
\begin{lstlisting}
quit = False
while not quit:
  action = raw_input("Should I quit?")
  if (action == "Yes") | (action == "yes"):
    quit = True
  else:
    print("You are not a quitter.")
\end{lstlisting}

\textbf{What does this code do?}

\textbf{How many steps does it take?}
\end{frame}

\begin{frame}[fragile]{Unbounded loop example}
\begin{lstlisting}[basicstyle=\ttfamily\tiny]
y = 10.0
vy = 0.0
dt = 0.05
while (abs(vy) > 0.9) | (y > 0.2):
  new_y = y + vy * dt
  if new_y <= 0.1:
    vy = -vy * 0.7
  else:
    vy -= 9.8 * dt
    y = new_y
  cls()
  screen = ""
  for j in range(0,20):
    for i in range(0,20):
      if (i == 10) & (j == 19 - int(y)):
        screen += "O"
      elif j == 19:
        screen += "-"
      else:
        screen += " "
    screen += "\n"
  print(screen)
  sleep(0.01)
\end{lstlisting}

\textbf{What does this code do?}

\textbf{How many steps does it take?}
\end{frame}

\begin{slide}{
\item \texttt{while} loops are very powerful
\item with great power comes...
\pause
\item ...greater chance of bugs
}\end{slide}

\begin{frame}[fragile]{Unbounded loop example}
\begin{lstlisting}[basicstyle=\ttfamily\tiny]
y = 10.0
vy = 0.0
dt = 0.05
while (abs(vy) > 0.9) | (y > 0.1):
  new_y = y + vy * dt
  if new_y <= 0.1:
    vy = -vy * 0.7
  else:
    vy -= 9.8 * dt
    y = new_y
  cls()
  screen = ""
  for j in range(0,20):
    for i in range(0,20):
      if (i == 10) & (j == 19 - int(y)):
        screen += "O"
      elif j == 19:
        screen += "-"
      else:
        screen += " "
    screen += "\n"
  print(screen)
  sleep(0.01)
\end{lstlisting}

\textbf{Does this loop terminate? (This is not the same code as before!)}

\pause

\textbf{No.} The condition has changed to \texttt{y > 0.1}.
\end{frame}


\begin{frame}[fragile]{Unbounded loop example}
\begin{lstlisting}[basicstyle=\ttfamily\tiny]
y = 10.0
vy = 0.0
dt = 0.1
while (abs(vy) > 0.9) | (y > 0.1):
  new_y = y + vy * dt
  if new_y <= 0.1:
    vy = -vy * 0.8
  else:
    vy -= 9.8 * dt
    y = new_y
  cls()
  screen = ""
  for j in range(0,20):
    for i in range(0,20):
      if (i == 10) & (j == 19 - int(y)):
        screen += "O"
      elif j == 19:
        screen += "-"
      else:
        screen += " "
    screen += "\n"
  print(screen)
  sleep(0.01)
\end{lstlisting}

\textbf{Does this loop terminate? (This is not the same code as before!)}

\pause

\textbf{No.} \texttt{dt = 0.1} and \texttt{vy = -vy * 0.8}.
\end{frame}

\SlideSection{Correctly encoding intentions}
\SlideSubSection{Why is \texttt{while} not enough}

\begin{slide}{
\item The expressive power of \texttt{while} is not always needed
\item Sometimes we want something simpler, and less dangerous
\item For example, consider:
\begin{itemize}
\item For each \textit{hostile alien}
\item Do \textit{attack it}
\end{itemize}
}\end{slide}

\begin{slide}{
\item A loop such as:
\begin{itemize}
\item For each \textit{hostile alien}
\item Do \textit{attack it}
\end{itemize}
\item Is predictable
\item Performs a fixed number of steps (one per hostile alien)
\item Will certainly terminate
}\end{slide}

\begin{slide}{
\item In general, we wish to always correctly encode our intention of repeating code $N$ times
\item The code must precisely fit our intentions, like a tailored italian suit
\begin{itemize}
\item Code should not be too complicated
\item Code should not be too simple
\end{itemize}
}\end{slide}

\SlideSubSection{Code that is too complicated?}
\begin{slide}{
\item A \texttt{while} loop where we need to perform \texttt{N} steps
\item There are many subtle ways to break the code
}\end{slide}

\begin{slide}{
\item Classes, objects, and inheritance everywhere
\item To know which code is actually run to say \texttt{Hello world!} you need to read twelve files
}\end{slide}

\begin{slide}{
\item Events, lambda's, higher-order combinators everywhere
\item To know what the program does you need two doctorates (CompSci and Maths)
\begin{itemize}
\item Plus internal access to the sliced brain of the original programmer
\end{itemize}
}\end{slide}

\SlideSubSection{Code that is too simple?}
\begin{slide}{
\item No handling of error cases
\item Ignoring hard circumstances
\pause
\item Not implementing all features correctly
\begin{itemize}
\item Showing progress off
\item Building impressive but pointless demo's
\end{itemize}
}\end{slide}

\SlideSection{Iterating with \texttt{for}}
\begin{slide}{
\item Python, and many other modern languages, offer explicit constructs for bounded repetition
\begin{itemize}
\item We specify precisely the number of steps that need to be performed
\item The language takes care of performing the right number of steps
\item The construct is much harder to break\footnote{Running forever} than a \texttt{while}-loop
\end{itemize}
\item These constructs are called for-loops
}\end{slide}

\SlideSubSection{Syntax of \texttt{for}}
\begin{slide}{
\item Number of repetitions (a \texttt{range} iterator)
\item That stores the index of the current repetition (a variable)
\item Body of the loop that is repeated at every iteration (a block of code)
}\end{slide}

\begin{frame}[fragile]{Syntax of \texttt{for}}
\begin{lstlisting}
for VARIABLE in range(END):
  BODY
\end{lstlisting}

\begin{itemize}
\item \texttt{VARIABLE} is any valid variable name that becomes useable within the \texttt{BODY}; will range from \texttt{0} to \texttt{END-1}
\item \texttt{END} is any positive number; the body will be repeated \texttt{END-1} times
\item \texttt{BODY} is a series of statements
\end{itemize}
\end{frame}

\SlideSubSection{Semantics of \texttt{for}}
%%%SEMANTICS

\SlideSubSection{Examples of \texttt{for}}


\begin{slide}{
\item Number of repetitions (a \texttt{range} iterator)
\item That stores the index of the current repetition (a variable)
\item Body of the loop that is repeated at every iteration (a block of code)
}\end{slide}


\begin{frame}{This is it!}
\center
\fontsize{18pt}{7.2}\selectfont
The best of luck, and thanks for the attention!
\end{frame}

\end{document}

\begin{slide}{
\item ...
}\end{slide}

\begin{frame}[fragile]
\begin{lstlisting}
...
\end{lstlisting}
\end{frame}

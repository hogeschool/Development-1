\documentclass{beamer}
\usetheme[hideothersubsections]{HRTheme}
\usepackage{beamerthemeHRTheme}
\usepackage[utf8]{inputenc}
\usepackage{graphicx}
\usepackage[space]{grffile}
\usepackage{listings}
\lstset{language=Python,
basicstyle=\ttfamily\footnotesize,
mathescape=true,
frame=single,
breaklines=true}
\lstset{
  literate={ï}{{\"i}}1
           {ì}{{\`i}}1
}
\usepackage{color}

\title{Looping with \texttt{for}}

\author{The INFDEV Team @ HR}

\institute{Hogeschool Rotterdam \\ 
Rotterdam, Netherlands}

\date{}

\begin{document}
\maketitle

\SlideSection{Introduction}
\SlideSubSection{Lecture topics}
\begin{slide}{
\item the (lack of) limitations of \texttt{while} loops
\item \texttt{for} statements and their semantics
\item \texttt{for} as a \textit{limited} form of \texttt{while}
}\end{slide}

\SlideSection{\texttt{while} loops}
\SlideSubSection{Potential issues}
\begin{slide}{
\item While loops specify unbounded iteration
\item This means that the number of iterations is not necessarily easy to specify
\item For example
\begin{itemize}
\item Virtual machines
\item User-driven loops
\item Servers
\item Operating systems
\item ...
\end{itemize}
}\end{slide}

\begin{frame}[fragile]{Unbounded loop example}
\begin{lstlisting}
n,m = input("Let's have two numbers")
cnt = 1
while n > m:
  cnt = cnt + 1
  n = n / m
print("Result is %d" % cnt)
\end{lstlisting}

\textbf{What does this code do?}

\textbf{How many steps does it take?}
\end{frame}

\begin{frame}[fragile]{Unbounded loop example}
\begin{lstlisting}
quit = False
while not quit:
  action = raw_input("Should I quit?")
  if (action == "Yes") | (action == "yes"):
    quit = True
  else:
    print("You are not a quitter.")
\end{lstlisting}

\textbf{What does this code do?}

\textbf{How many steps does it take?}
\end{frame}

\begin{frame}[fragile]{Unbounded loop example}
\label{slide:bouncing}

\begin{lstlisting}[basicstyle=\ttfamily\tiny]
y = 10.0
vy = 0.0
dt = 0.05
while (abs(vy) > 0.9) | (y > 0.2):
  new_y = y + vy * dt
  if new_y <= 0.1:
    vy = -vy * 0.7
  else:
    vy -= 9.8 * dt
    y = new_y
  ... draw a ball at position (10,y) ...
\end{lstlisting}

\textbf{What does this code do?}

\url{https://github.com/hogeschool/INFDEV01-1/blob/master/code/bouncing%20ball%20sample/bouncing%20ball.py}

\textbf{How many steps does it take?}
\end{frame}

\begin{slide}{
\item \texttt{while} loops are very powerful
\item with great power comes...
\pause
\item ...greater chance of bugs
}\end{slide}

\begin{slide}{
\item Subtle changes might affect behaviour deeply
\item For example, a change in value of \texttt{0.1} makes the loop non-terminating
\item The culprit may be hidden in a lot of places
\begin{itemize}
\item Floating point errors
\item Logical repetition: state always changes, within a circular trajectory
\item ...
\end{itemize}
}\end{slide}

\begin{frame}[fragile]{Unbounded loop example}
\begin{lstlisting}[basicstyle=\ttfamily\tiny]
y = 10.0
vy = 0.0
dt = 0.05
while (abs(vy) > 0.9) | (y > 0.1):
  new_y = y + vy * dt
  if new_y <= 0.1:
    vy = -vy * 0.7
  else:
    vy -= 9.8 * dt
    y = new_y
  ... draw a ball at position (10,y) ...
\end{lstlisting}

\textbf{Does this loop terminate? (This is not the same code as in Slide \ref{slide:bouncing}!)}

\pause

\textbf{No.} The condition has changed to \texttt{y > 0.1}.
\end{frame}


\begin{frame}[fragile]{Unbounded loop example}
\begin{lstlisting}[basicstyle=\ttfamily\tiny]
y = 10.0
vy = 0.0
dt = 0.1
while (abs(vy) > 0.9) | (y > 0.2):
  new_y = y + vy * dt
  if new_y <= 0.1:
    vy = -vy * 0.8
  else:
    vy -= 9.8 * dt
    y = new_y
  ... draw a ball at position (10,y) ...
\end{lstlisting}

\textbf{Does this loop terminate? (This is not the same code as in Slide \ref{slide:bouncing}!)}

\pause

\textbf{No.} \texttt{dt = 0.1} and \texttt{vy = -vy * 0.8}.
\end{frame}

\SlideSection{Correctly encoding intentions}
\SlideSubSection{Why is \texttt{while} not enough}

\begin{slide}{
\item The expressive power of \texttt{while} is not always needed
\item Sometimes we want something simpler, and less dangerous
\item For example, consider:
\begin{itemize}
\item For each \textit{hostile alien}
\item Do \textit{attack it}
\end{itemize}
}\end{slide}

\begin{slide}{
\item A loop such as:
\begin{itemize}
\item For each \textit{hostile alien}
\item Do \textit{attack it}
\end{itemize}
\item Is predictable
\item Performs a fixed number of steps (one per hostile alien)
\item Will certainly terminate
}\end{slide}

\begin{slide}{
\item In general, we wish to always correctly encode our intention of repeating code $N$ times
\item The code must precisely fit our intentions, like a tailored italian suit
\begin{itemize}
\item Code should not be too complicated
\item Code should not be too simple
\end{itemize}
}\end{slide}

\SlideSubSection{Code that is too complicated?}
\begin{slide}{
\item A \texttt{while} loop where we need to perform \texttt{N} steps
\item There are many subtle ways to break the code
}\end{slide}

\begin{slide}{
\item Classes, objects, and inheritance everywhere
\item To know which code is actually run to say \texttt{Hello world!} you need to read twelve files
}\end{slide}

\begin{slide}{
\item Events, lambda's, higher-order combinators everywhere
\item To know what the program does you need two doctorates (CompSci and Maths)
\begin{itemize}
\item Plus internal access to the sliced brain of the original programmer
\end{itemize}
}\end{slide}

\SlideSubSection{Code that is too simple?}
\begin{slide}{
\item No handling of error cases
\item Ignoring hard circumstances
\pause
\item Not implementing all features correctly
\begin{itemize}
\item Showing progress off
\item Building impressive but pointless demo's
\end{itemize}
}\end{slide}

\SlideSection{Iterating with \texttt{for}}
\begin{slide}{
\item Python, and many other modern languages, offer explicit constructs for bounded repetition
\begin{itemize}
\item We specify precisely the number of steps that need to be performed
\item The language takes care of performing the right number of steps
\item The construct is much harder to break\footnote{Running forever} than a \texttt{while}-loop
\end{itemize}
\item These constructs are called for-loops
}\end{slide}

\SlideSubSection{Syntax of \texttt{for}}
\begin{slide}{
\item Number of repetitions (a \texttt{range} iterator)
\item That stores the index of the current repetition (a variable)
\item Body of the loop that is repeated at every iteration (a block of code)
}\end{slide}

\begin{frame}[fragile]{Syntax of \texttt{for}}
\begin{lstlisting}
for VARIABLE in range(END):
  BODY
\end{lstlisting}

\begin{itemize}
\item \texttt{VARIABLE} is any valid variable name that becomes useable within the \texttt{BODY}; will range from \texttt{0} to \texttt{END-1}
\item \texttt{END} is any positive number; the body will be repeated \texttt{END-1} times
\item \texttt{BODY} is a series of statements
\end{itemize}
\end{frame}

\SlideSubSection{Semantics of \texttt{for}}
\begin{slide}{
\item The general form is \texttt{for VAR in RANGE: BODY} ($f_{VRB}$)
\item If \texttt{VAR} is still within \texttt{RANGE}, then we jump to the beginning of \texttt{BODY} and then increment the variable, otherwise we jump to the end of the whole \texttt{for}
$$
\tiny
\left\{
\begin{matrix}
(PC,S) \overset{f_{VRB}}{\rightarrow} (firstLine(B),S) &when& S[V] \in R & \\
(PC,S) \overset{f_{VRB}}{\rightarrow} (skipAfter(B),S) &when& S[V] \notin R & \\
\end{matrix}
\right.
$$
\item At the end of the loop assume that we have two invisible instructions
\begin{itemize}
\item \texttt{V = V + 1}
\item \texttt{jump back to begin loop}
\end{itemize}
}\end{slide}

\SlideSubSection{Index of the current repetition}
\begin{slide}{
\item The \texttt{BODY} of the \texttt{for} loop is always the same
\item Depending on the current step, we may perform different processing
\item For this, we need to know how far we have come in the loop
\begin{itemize}
\item The iteration \texttt{VARIABLE} tells us this
\end{itemize}
}\end{slide}

\begin{frame}[fragile]{Index of the current repetition}
\begin{lstlisting}
graph = ""
for i in range(11):
  for j in range((i-5) * (i-5)):
    graph += '='
  graph += "\n"
print(graph)
\end{lstlisting}

\textbf{What does this do?}

\textbf{How do the different steps perform different actions based on the value of the iteration variable(s)?}
\end{frame}

\begin{slide}{
\item Different processing per different steps makes the loop perform a more complex operation.
\item Complex is not the same as complicated.
\item To avoid needless complication, the different steps must still do related things
}\end{slide}

\begin{frame}[fragile]{Unrelated actions in a loop}
\begin{lstlisting}
for i in range(6):
  if i == 0:
    #...run a game of tic-tac-toe...
  elif i == 1:
    #...draw a smiley...
  elif i == 2:
    #...run a turtle program...
  elif i == 3:
    #...convert degrees to fahrenheit...
  elif i == 4:
    #...draw a square...
  elif i == 5:
    #...draw a triangle...
\end{lstlisting}

\textbf{What is the relationship between the iterations?}

\textbf{Is a \texttt{for} loop really needed?}
\end{frame}

\SlideSubSection{Body of the loop}
\begin{slide}{
\item The code of the body is a block of code
\item A block of code is any statement or series of statements
\item Among these statements, we can use as many \texttt{if}'s, \texttt{for}'s, and \texttt{while}'s
}\end{slide}

\begin{slide}{
\item There is no \textit{obfuscated code} prize available
\item Nesting too many complex constructs might make code \textbf{needlessly complicated}
\item Remember that a \texttt{for}-loop adds a large number of possible execution paths, just like a \texttt{while}-loop
}\end{slide}

\SlideSection{Advanced loops}
\SlideSubSection{Ranges of iteration}
\begin{slide}{
\item We do not always want to iterate through values between \texttt{0} and a given number
\item Even though we still need to perform a fixed number of steps
\item So a \texttt{for}-loop still has advantages over a \texttt{while}-loop
}\end{slide}

\SlideSubSection{We might want to...}
\begin{slide}{
\item ...decrement instead of increment, that is ``go backwards''
\item ...iterate between a range of values that does not start with zero
\item ...take steps of more than one between iterations
}\end{slide}

\SlideSubSection{The \texttt{range} function}
\begin{slide}{
\item Actually takes three parameters: \texttt{range(start, end, step)}
\item With one parameter we only specify \texttt{end}, while \texttt{start = 0} and \texttt{step = 1}
\item With two parameters we specify \texttt{start} and \texttt{end}, while \texttt{step = 1}
\item With all parameters we specify \texttt{start}, \texttt{end}, and \texttt{step}
}\end{slide}

\begin{frame}[fragile]{Specific starting point}
\begin{lstlisting}
for i in range(2, 10, 1):
  print(i)
\end{lstlisting}
\end{frame}

\begin{frame}[fragile]{Multiple steps}
\begin{lstlisting}
for i in range(0, 20, 5):
  print(i)
\end{lstlisting}
\end{frame}

\begin{frame}[fragile]{Backwards range}
\begin{lstlisting}
for i in range(10, 0, -1):
  print("           \r" + str(i)),
  sleep(0.3)
print("\rBOOOM!!!!!")
\end{lstlisting}
\end{frame}

\SlideSubSection{Nesting \texttt{for}-loops}
\begin{slide}{
\item The \texttt{BODY} of a \texttt{for}-loop contains arbitrary code
\item This arbitrary code may also contain loops
\item Loops within loops have a ``multiplicative'' behaviour
\begin{itemize}
\item A loop of \texttt{M} step within a loop of \texttt{N} steps performs \texttt{N*M} steps
\end{itemize}
}\end{slide}

\begin{frame}[fragile]{Multiplicative behaviour}
\begin{lstlisting}
cnt = 0
for i in range(10):
  for j in range(5):
    cnt += 1
print(cnt)
\end{lstlisting}
\end{frame}

\begin{slide}{
\item Each loop adds its own iteration variable
\item The iteration variables, together, are an N-dimensional point
\item A single loop performs a ``linear'' computation, two loops perform a ``square''     computation, three perform a ``cubic'' computation, etc.
}\end{slide}

\begin{slide}{
\item Multiple \texttt{for}-loops perform a predetermined number of computations
\item This means that we can always translate multiple \texttt{for}-loops into a single one\footnote{This will usually break readability, so it is not advised: it is just a reasoning exercise.}
}\end{slide}

\begin{frame}[fragile]
\begin{lstlisting}
for i in range(0,10):
  for j in range(0,5):
    print(i,j)
\end{lstlisting}

\textbf{can be simulated with}

\begin{lstlisting}
for x in range(0,50):
  i = x / 5
  j = x % 5
  print(i, j)
\end{lstlisting}
\end{frame}


\SlideSection{Conclusion}
\SlideSubSection{Conclusion}
\begin{slide}{
\item \texttt{while}-loops can encode any form of iteration.
\item When the number of iterations is known beforehand, while is too powerful
\item To use the right level of abstraction (which is less sensitive to bugs), we use \texttt{for}-loops instead
\item This allows us to instruct the language to perform exactly the required number of steps, usually with less code
}\end{slide}

\begin{frame}{This is it!}
\center
\fontsize{18pt}{7.2}\selectfont
The best of luck, and thanks for the attention!
\end{frame}

\end{document}

\begin{slide}{
\item ...
}\end{slide}

\begin{frame}[fragile]
\begin{lstlisting}
...
\end{lstlisting}
\end{frame}
